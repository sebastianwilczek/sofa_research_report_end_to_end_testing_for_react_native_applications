\section{Conclusion}
\label{sec:Conclusion}
\lhead{Conclusion}

As part of this research, many testing frameworks were considered whether they are easy and efficient to test with and if their usage is applicable in the project in question. According to the interviews conducted with \textit{React Native} developers working on the project, this means that the framework should ideally be cross-platform compatible both for the mobile OS, as in \textit{Android} and \textit{iOS}, although \textit{Android} is the preferred platform, as well as the development platform, meaning \textit{Windows}, \textit{Linux} and \textit{mac OS}. Furthermore, the framework should require as little change to the source code as possible, both to existing code as well as how code should be written in the future of the project.
\newline
All examined platforms have their advantages and disadvantages. However, the research results in the conclusion that, at least for the project in question, none of the testing frameworks can be applied and used in an efficient way.
\newline
\textit{Jest} offers great functionality when it comes to testing singular components. The snapshot feature makes it test at a very high speed without the need for any emulation, and it's platform independence makes it usable almost everywhere in the context of \textit{React Native}. However, the framework was designed with a more component-based testing approach in mind, and it is best used that way. The simulation of user input and information processing can not be created in a test environment using only \textit{Jest}.
\newline
\textit{Appium} has the advantage of having a great interaction API and tests that are easily readable. It is also able to run cross-platform. However, the dependency on \textit{Accessibility Labels} makes it incompatible with some \textit{React Native} components, as it is the case with the already developed product. Furthermore, using the labels is more change to the source code and also indirectly perceivable by the user of the application.
\newline
\textit{Espresso} does have a good approach with it's synchronisation. However, this feature is also available in other frameworks, for instance \textit{Detox}, and it might not be the most up-to-date framework available. It's limitation to the Android platform also makes it less than ideal.
\newline
\textit{Detox} combines the easy usability of \textit{Appium} and the great synchronisation features of \textit{Espresso}. It is also simple to set up, offering the possibility of defining many different testing configurations. However, the API is more limited than the one of \textit{Appium} and, same as \textit{Appium}, components are not recognised during the testing procedure. For now, the functionality is also limited when it comes to the \textit{Android} platform.
\newline
Lastly, \textit{Cavy} has a great approach when referencing components when one thinks about performance, since the references used by \textit{Cavy} are only making use of the \textit{React Native} framework, instead of native platform identification. However, just like \textit{Appium} and \textit{Detox}, this requires every component to be marked as a testable component during development, making the selection API very limited, with the added disadvantage of needing to wrap the entire application in a test hooking function. Furthermore, \textit{Cavy} does not support functional components, another paradigm that is made heavily use of in the project in question.
\newline
For the reason reiterated above, no testing framework is completely applicable to the project. However, if one were to create a new \textit{React Native} application, the following suggestions could be made:
\newline
Every major component should be unit or component tested making use of \textit{Jest}'s snapshot feature. For end-to-end testing, developers should decide between \textit{Appium} and \textit{Detox}, depending on the needed complexity of the written tests, with \textit{Appium} for more complex tests. If the required complexity level of tests is not as high, \textit{Detox} can be used for the sake of an easy setup and configuration. In any case, it should be paid attention to the fact that the components that are made use of need to be able to be referenced by the chosen test framework, for example through \textit{Accessibility Labels}. If it can be made sure that no functional components will ever be used as part of the application, \textit{Cavy} may also be used as an alternative.
\newline
Overall, each testing framework does what it is intended to do, which is testing a \textit{React Native} application. This research has returns some of the caveats and disadvantages of the tested frameworks. However, it also shows some of the positive aspects of each as well as potential reasons why and environments to use them in. One thing that can be taken away from this research is that testing is a part of software development that should be considered from the very beginning of a project, even when working with a very frontend focused framework such as \textit{React Native}. Introducing tests late in a project results mostly in errors and compatibility issues, like it did in this project. If tests are written and executed properly from the start, both the product and the tests written for it can develop into software that works as intended.