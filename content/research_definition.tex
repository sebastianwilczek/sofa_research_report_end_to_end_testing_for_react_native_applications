\section{Research Definition}
\label{sec:research_definition}
\lhead{\thesection \space Research Definition}

This chapter contains details as to how the research was conducted. This includes an explanation as to why the research was necessary in the first place as well as a set of questions that the research aims to answer. Furthermore, it is detailed what kind of research methods were made use of during the research.

\subsection{Research Motivation}
\label{ssec:research_motivation}

The research is part of a project on the \textit{Connected.Football} mobile application. The application itself is developed using \textit{React Native}, a frontend framework developed by Facebook (\cite{react-native}).
\newline
During development, the development team realised that they do a lot of manual testing, as in continuously going through the same repetitive steps to test certain use cases, for example entering an email and a password every time they want to test something that requires a user to be logged in. Eventually, the question came up as to how this form of testing can be done more efficiently, since the developers lose a huge amount of time testing repetitive interactions. Furthermore, due to the repetitive nature of such tests, the developers may overlook some flaws in the developed product, which could also be avoided through proper tests.

\subsection{Research Questions}
\label{ssec:research_questions}

As a guideline of what should be researched, the following research questions were defined:

\begin{enumerate}
\item How can React Native applications be end-to-end tested?
\item What framework is the best to use for the React Native application in question?
\begin{enumerate}
\item Does it fit the development environment?
\item Is it easy to use, in terms of setup, test development and execution?
\item Is it compatible with the already existing product?
\end{enumerate}
\item Is end-to-end testing a feasible thing to do in the project in question?
\end{enumerate}

\subsection{Research Methods}
\label{ssec:research_methods}

To answer the aforementioned questions and to conduct the research, a set of methods is to be made use during the research. First of all, the research conduction will include mostly literature research. It will be researched what frameworks are available and what the advantages and disadvantages to using them are. Literature to be used comes from the official documentations of the various frameworks as well as by publications of developers making use of specific frameworks.
\newline
Furthermore, interviews with the \textit{React Native} developers of the project in question are to be conducted. The interview topics will be how testing is currently done in development, how it could be made more efficient, what their development environment is like when it comes to for instance operating systems and whether or not existing code should be changed or more work put into future development for the sake of testing.

\subsection{Interview Questions}
\label{ssec:interview_questions}

As mentioned in \textit{Chapter \ref{ssec:research_methods}}, an interview with \textit{React Native} developers is part of the research methods applied. The following questions are to be asked to the developers and answers to said questions will be referenced in the coming chapter.

\begin{enumerate}
\item How do you currently test artefacts developed in \textit{React Native}?
\item Does the testing include any repetitive or unnecessary elements?
\item Could the testing process be made more efficient? If yes, how?
\item What is included in your development environment? (OS, Mobile OS, emulation, physical devices)
\item Is it worth it to change existing code or to put extra work into future work on source code, if tests can be made more accessible and easier to use this way?
\end{enumerate}

A transcript of all interviews can be found in the Appendix (see \textit{Appendices \ref{appendix:interview_lucas_gehlen}, \ref{appendix:interview_marco_kull}, \ref{appendix:interview_patrick_richter}}).