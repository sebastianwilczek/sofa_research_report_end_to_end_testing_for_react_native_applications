\section{Introduction}
\label{sec:introduction}
\lhead{\thesection \space Introduction}

This document is a research report on the topic of end-to-end testing of \textit{React Native} applications. \textit{React Native} is a framework developed by Facebook that enables developers to write one singular base of \textit{JavaScript} development code for a mobile based application and to then deploy them to various mobile platforms, predominately \textit{Android} and \textit{iOS}. The framework itself is based on \textit{ReactJS} and is handled similarly (\cite{react-native}).
\newline
The research was conducted as part of a development project involving the mobile application \textit{Connected.Football}, which is being developed using \textit{React Native}. During development, the question came up as to how such an application can be properly tested, especially how it can be assured that a user of the application is able to use it as intended.
\newline
The following chapters will first of all detail how the research was planned to be conducted. To illustrate that, it will be defined why the research was done in the first place, what questions were asked and answered, and what methods were to be made use of during the research.
\newline
What follows are the results of the research. The results contain mostly a collection of different testing frameworks and their advantages when compared to others. Each framework is tested regarding the requirements necessary, as they are defined by the project work and the environment of the \textit{Connected.Football} application. The result chapter also contains sample code written for some of the researched framework. Furthermore, it is explained how the frameworks could be used in the application development in question as well as in general \textit{React Native} development.
\newline
The report ends with a conclusion, summarising the results of the research and giving advice as to how the testing frameworks are to be used in the project. It also contains general advice as to how to approach \textit{React Native} end-to-end testing.